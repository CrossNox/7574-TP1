\documentclass{article}
\usepackage[utf8]{inputenc}
\usepackage{amsmath}
\usepackage{mathabx}
\usepackage{graphicx}
\usepackage{minted}
\usepackage{booktabs}
\usepackage[english,spanish,es-noindentfirst,es-nosectiondot,es-nolists,
es-noshorthands,es-lcroman,es-tabla]{babel}
\usepackage{lmodern}             % Use Latin Modern fonts
\usepackage[T1]{fontenc}         % Better output when a diacritic/accent is used
\usepackage[utf8]{inputenc}      % Allows to input accented characters
\usepackage{textcomp}            % Avoid conflicts with siunitx and microtype
\usepackage{microtype}           % Improves justification and typography
\usepackage[svgnames]{xcolor}    % Svgnames option loads navy (blue) colour
\usepackage[hidelinks,urlcolor=blue]{hyperref}
\hypersetup{colorlinks=true, allcolors=Navy, pdfstartview={XYZ null null 1}}
\newtheorem{lemma}{Lema}
\usepackage[width=14cm,left=3.5cm,marginparwidth=3cm,marginparsep=0.35cm,
height=21cm,top=3.7cm,headsep=1cm, headheight=1.6cm,footskip=1.2cm]{geometry}
\usepackage{csquotes}
\usepackage{biblatex}
\addbibresource{informe.bib}

\title{Distribuidos I \- TP1}
\author{Mermet, Ignacio Javier \texttt{98153}}
\date{Abril 2022}

%TODO: mejorar carátula

\begin{document}

\maketitle

\section{Sobre la entrega}
El código de la entrega se puede encontrar en \href{https://github.com/CrossNox/7574-TP1}{GitHub}.

\section{Estructura del proyecto}
El proyecto fue desarrollado en \texttt{python}\cite{Python} y empaquetado con \texttt{poetry}\cite{PythonPoetry}. Tiene dos CLIs asociadas:

\begin{itemize}
	\item \texttt{metrics_server}: el servidor de métricas
	\item \texttt{metrics_client}: cliente para enviar, consultar y monitorear métricas
\end{itemize}

En la carpeta \texttt{docker} se encuentran disponibles los \texttt{Dockerfile} asociados tanto al servidor como al cliente.

\section{Protocolo de comunicación}
La comunicación cliente-servidor utiliza un protocolo binario. La comunicación se inicia enviando un paquete \texttt{Intention}\cite{IntentionPackage} que indica que tipo de operación se desea ejecutar:
\begin{itemize}
	\item Enviar métrica
	\item Consultar métrica
	\item Monitorear notificaciones
\end{itemize}

Luego el protocolo de comunicación continúa de acuerdo a cada caso particular.

\subsection{Envío de métrica}

\subsection{Consulta de métrica}

\subsection{Monitoreo de notificaciones}

\section{Arquitectura general}
\subsection{Diagrama de robustez}

\subsection{Diagrama de clases}

\subsection{Diagrama de paquetes}

\subsection{Vista física}

\subsection{Vista de procesos}

\section{Resolución de concurrencia}

\subsection{Concurrencia entre lectura y escritura de archivos}

\printbibliography

\end{document}
